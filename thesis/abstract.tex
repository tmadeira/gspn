\chapter*{Abstract}

MADEIRA, Tiago. \textbf{\@title}. Thesis (Masters). Institute of Mathematics and Statistics, University of São Paulo, São Paulo, 2023.

\vspace{1em}

This thesis is about finding maxima of Sum-Product Networks (SPNs). SPNs are expressive statistical deep models that efficiently represent complex probability distributions. They encode context-specific independence among random variables and enable exact marginal and conditional probability inference in linear time.

The research explores Gaussian SPNs (GSPNs), which are continuous SPNs with Gaussian distributions at their leaves. GSPNs provide compact representations of Gaussian Mixture Models (GMMs) with many components. The relationship between GSPNs and GMMs has been largely unexplored in the literature, particularly regarding mode-finding techniques. The problem of finding modes in Gaussian mixtures is challenging, and existing techniques involve hill-climbing algorithms. However, there is limited research discussing modes in the context of SPNs.

The objective of this work is to investigate and establish a framework for identifying modes in GSPNs. This is accomplished by developing an algorithm that employs an EM-style fixed-point iteration method for mode finding in GSPNs. The algorithm is presented in detail, accompanied by a formal proof of its correctness. Two applications for it are discussed: Maximum-A-Posteriori inference and modal clustering. Some experimental results are provided to evaluate the effectiveness of the proposed approach.

\vspace{1em}

\noindent \textbf{Keywords:} Sum-Product Networks; Gaussian Mixture Models; Mode Finding; Probabilistic Models; Machine Learning.
