\chapter*{Resumo}

MADEIRA, Tiago. \textbf{Encontrando Máximos de Redes Soma-Produto Gaussianas}. Dissertação (Mestrado). Instituto de Matemática e Estatística, Universidade de São Paulo, São Paulo, 2023.

\vspace{1em}

Esta dissertação é sobre busca de máximos de Redes Soma-Produto (SPNs, do inglês Sum-Product Networks). As SPNs são modelos estatísticos profundos expressivos que representam eficientemente distribuições de probabilidade complexas. Elas codificam independência contextual específica entre variáveis aleatórias e permitem inferência exata de probabilidade marginal e condicional em tempo linear.

A pesquisa explora as SPNs Gaussianas (GSPNs), que são SPNs contínuas com distribuições Gaussianas em suas folhas. As GSPNs fornecem representações compactas de Modelos de Misturas Gaussianas (GMMs) com muitos componentes. A relação entre GSPNs e GMMs tem sido pouco explorada na literatura, especialmente no que diz respeito a técnicas de busca de modas. O problema de encontrar modas em misturas Gaussianas é desafiador e as técnicas existentes envolvem algoritmos de escalada. No entanto, há pouca pesquisa discutindo modas no contexto de SPNs.

O objetivo deste trabalho é investigar e estabelecer uma abordagem para encontrar modas em GSPNs. Isso é alcançado através do desenvolvimento de um algoritmo que utiliza um método de iteração de ponto fixo no estilo EM (Expectativa-Maximização) para encontrar modas em GSPNs. O algoritmo é apresentado em detalhes, acompanhado de uma prova formal de sua corretude. Duas aplicações para ele são discutidas: inferência de Máximo-A-Posteriori e clusterização modal. Alguns resultados experimentais são fornecidos para avaliar a eficácia da abordagem proposta.

\vspace{1em}

\noindent \textbf{Palavras-chave:} Redes Soma-Produto; Modelos de Misturas Gaussianas; Busca de Modas; Modelos Probabilísticos; Aprendizagem de Máquina.
